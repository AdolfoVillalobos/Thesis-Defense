
\section{Conclusions}
\frame{\sectionpage}

\begin{frame}{Conclusions}
    \begin{alertblock}{Findings}
        We showed that using Low-Cost IMUS and a data-driven approach:
        \begin{enumerate}
            \item Is possible to detect the type cut that the worker is performing.
            \item We developed an accurate classifier to predict the presence of risk factors due to hyperextensions.
            \item We can detect when the knife has lost his sharpness.
        \end{enumerate}
    \end{alertblock}
\end{frame}


\begin{frame}{Conclusions}
    \begin{alertblock}[Limitations]
        There are limitations
        \begin{enumerate}
        \item Accuracy in cutting-task type is not high enough.
        \item Deterioration of knife is discrete, not continuous.
        \item We need supervision of the beginning and end of each cutting task. 
        \end{enumerate}
    \end{alertblock}
\end{frame}

\begin{frame}{Conclusions}
    \begin{alertblock}[Future Work]
        Future work:
        \begin{enumerate}
            \item Algorithms to determine when worker is performing a task or not.
            \item Use IMU for retraining purposes.
            \item Develop a DSS framework.
            \item Determine the productivity of the collaborator
            \item Build a production-ready version of the prototype.
        \end{enumerate}
    \end{alertblock}
\end{frame}